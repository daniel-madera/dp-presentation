

\input{settings_FM_template.tex}


\usepackage{palatino}
\usepackage{graphicx}
\usepackage{transparent}

\setbeamercovered{transparent=35}

\title[Pokročilý třídní jazykový slovníček]{Pokročilý třídní jazykový slovníček}
\subtitle{Diplomová práce}
\author[Bc.~Daniel~Maděra]{Bc.~Daniel~Maděra}
\institute[TUL]{Technická univerzita v Liberci}
\date{1.~února~2017}
\newcommand{\TextTitulniStranaPodLinkou}{\tiny
Studentská 2 {\color{FM_TUL} |} 461\,17 Liberec 2 {\color{FM_TUL} |} {daniel.madera@tul.cz} {\color{FM_TUL} |} 
\href{http://www.fm.tul.cz/}{www.fm.tul.cz}}

\renewcommand{\inserttotalframenumber}{\pageref{lastslide}}

\begin{document}
%\setbeamertemplate{caption}{\insertcaption}


\begin{frame}
    \titlepage
\end{frame}

\begin{frame}
    \frametitle{Obsah}
    \begin{itemize}
        \item problematika
            % - učení slovíček při hodinách -> vlastní seznamy slovíček
            % - ostatní řešení žákům neposkytují přizpůsobení úrovně ani obsahu
            % - cílem diplomové práce je provést analýzu, jak zefektivnit učení a motivovat žáky k procvičování slovíček 
            % - na základě zjištěných poznatků navrhnout a implementovat aplikaci, která umožní rozvíjet cizojazyčnou slovní zásobu % - a zároveň vyučujícím poskytne nástroj pro správu a prezentaci slovíček pro žáky
        \item představení aplikace
            % - koncept (funkcionalita)
            %     + každý může tvořit
            % - nástroj pro vyučující 
            %     + tvorba učebnic (správa slovíček, modulů)
            %         - import slov
            %         - sdílení učebnic
            %     + tvorba tříd (oddělené skupiny)
            % - nástroj pro studenty
            %     + procvičování slovíček z učebnic i vlastních
            %     + připomínání slovíček v narůstajících intervalech
            %         - permanentní zapamatování
            % - automatizované poskytnutí zvukové, obrazové a textové formy
            % - snaha o motivaci
            % zvuková forma - syntéza pomocí Google Text To Speeach API
        \item algoritmizace
            % - metoda rozloženého opakování
            %     - supermemo
            %     - adaptovaný leiterův algoritmus (obrázek)
            %         - 2 fáze (zjištění úrovně žáka, samotné procvičování)
            % - aktivní vzpomínání a rozpoznání slova
            %     - algoritmus
        \item technologie
        % \item grafické rozhraní
        \item shrnutí
    \end{itemize}
\end{frame}


\begin{frame}[t]
    \frametitle{Problematika}
    \begin{block}{Učení slovníček cizího jazyka}
    \begin{itemize}[<+->]
        \item systém vlastního slovníčku
        % neorganizované, 
        % Žáci si při výuce cizího jazyka obvykle vedou vlastní slovníček, do kterého si zapisují nově nabytá slova. V době přípravy na test rychle slova zopakují a následně z krátkodobé paměti test napíšou, což má za následek, že pokud naučená slova dále neopakují, dojde k jejich zapomenutí. Navíc se postupně slovíčka hromadí a není v silách žáka je opakovat.
        \item výukové aplikace
    \end{itemize}
    \end{block}
    
    \begin{block}{Výukové nástroje a aplikace}
    \begin{itemize}[<+->]
        \item neumožňují definici vlastních slov
        \item přizpůsobit zásobu aktuálně probírané
        \item přizpůsobení úrovni žáka
    \end{itemize}
    \end{block}
\end{frame}

\begin{frame}[c]
    \frametitle{Cíle práce}
    \begin{itemize}[<+->]
        \item vytvořit výukový nástroj
    \end{itemize}
    
    \begin{block}{Pro vyučující}
    \begin{itemize}[<+->]
        \item správa učebnic se slovíčky
        \item zpřístupnění slovíček žákům
    \end{itemize}
    \end{block}

    \begin{block}{Pro žáky}
        \begin{itemize}[<+->]
            \item naučit a procvičovat slova
            \item připomínat naučená slova
            \item zvuková, textová a obrazová forma
            \item motivace
        \end{itemize}
    \end{block}
\end{frame}

\begin{frame}[t]
    \frametitle{Hlavní myšlenky aplikace}
    % možná obrázek místo textu
    \begin{itemize}[<+->]
        \item nejen pro vyučující a žáky (kdokoliv)
        \item sloužit jako portál
        \item sdílení učebnic
        \item tvorba testovacích sad
        \item účelem není testovat žáky
        % nemá za úkol zkoušet slovíčka, spíše usnadnit přípravu žáků na následující hodinu a zároveň rozvíjet slovní zásobu
        \item efektivně rozvíjet slovní zásobu dle probírané učebnice
        % rozvíjet slovní zásobu
    \end{itemize}
\end{frame}

\begin{frame}[t]
    \frametitle{Generování slov v procvičování - inicializace}
    \only<1> {\begin{center}\includegraphics[width=8cm]{./img/words-gen-algorithm.png}\end{center}}
    \only<2> {\begin{center}\includegraphics[width=8cm]{./img/words-gen-algorithm-0.png}\end{center}}
    \only<3> {\begin{center}\includegraphics[width=8cm]{./img/words-gen-algorithm-1.png}\end{center}}
    \only<4> {\begin{center}\includegraphics[width=8cm]{./img/words-gen-algorithm-2.png}\end{center}}
    \only<5> {\begin{center}\includegraphics[width=8cm]{./img/words-gen-algorithm-3.png}\end{center}}
    \only<6> {\begin{center}\includegraphics[width=8cm]{./img/words-gen-algorithm-4.png}\end{center}}
    \only<7> {\begin{center}\includegraphics[width=8cm]{./img/words-gen-algorithm-5.png}\end{center}}
    \only<8> {\begin{center}\includegraphics[width=8cm]{./img/words-gen-algorithm-6.png}\end{center}}
\end{frame}

\begin{frame}[fragile]
    \frametitle{Generování slov v procvičování - testování}
    \begin{columns}[T] % align columns
        \begin{column}{.48\textwidth}
            \only<1> {\begin{center}\includegraphics[width=\textwidth]{./img/words-gen-algorithm-leitner.png}\end{center}}
            \only<2> {\begin{center}\includegraphics[width=\textwidth]{./img/words-gen-algorithm-leitner-1.png}\end{center}}
            \only<3> {\begin{center}\includegraphics[width=\textwidth]{./img/words-gen-algorithm-leitner-2.png}\end{center}}
            \only<4> {\begin{center}\includegraphics[width=\textwidth]{./img/words-gen-algorithm-leitner-3.png}\end{center}}
            \only<5> {\begin{center}\includegraphics[width=\textwidth]{./img/words-gen-algorithm-leitner-4.png}\end{center}}
            \only<6-> {\begin{center}\includegraphics[width=\textwidth]{./img/words-gen-algorithm-leitner-5.png}\end{center}}
        \end{column}%
        \hfill%
        \begin{column}{.48\textwidth}
            \begin{block}{Testování}
                \begin{itemize}
                    \item<2-> 1. průchod
                    \item<3-> 2. průchod
                    \item<4-> 3. průchod
                    \item<5-> správná a neúplná odpověď
                    \item<6-> špatná odpověď
                    \item<7-> konec: všechna slova jsou dokončená
                \end{itemize}
            \end{block}
        \end{column}%
    \end{columns}
\end{frame}

\begin{frame}[t]
    \frametitle{Připomínání slov}
    \begin{itemize}
        \item metoda rozloženého opakování (Supermemo)
        \item základ pro efektivní učení slov
    \end{itemize}
    \begin{center}
        \includegraphics[width=6.5cm]{./img/forgetting-curve.pdf}\\
    \end{center}
\end{frame}

\begin{frame}
\begin{center}
\label{lastslide}
\huge Děkuji za pozornost.
\end{center}
\end{frame}

\begin{frame}[noframenumbering]
\begin{center}
    \frametitle{Existují další možnosti stanovení obtížnosti slovíček? Jakým způsobem by změna metody ovlivnila aplikaci?}
    
\end{center}
\end{frame}

\begin{frame}[noframenumbering]
\begin{center}
    \frametitle{Jaký může být důsledek toho, že není použito API dle REST ale vlastní řešení?}
    
\end{center}
\end{frame}

\begin{frame}[noframenumbering]
\begin{center}
    \frametitle{Pro práci byla zvolena integrace se službami Google. Byly zvažovány i další možnosti, např. integrace výkladového slovníku apod.?}
    
\end{center}
\end{frame}

\end{document}
